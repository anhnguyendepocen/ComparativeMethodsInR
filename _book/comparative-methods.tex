\documentclass[]{book}
\usepackage{lmodern}
\usepackage{amssymb,amsmath}
\usepackage{ifxetex,ifluatex}
\usepackage{fixltx2e} % provides \textsubscript
\ifnum 0\ifxetex 1\fi\ifluatex 1\fi=0 % if pdftex
  \usepackage[T1]{fontenc}
  \usepackage[utf8]{inputenc}
\else % if luatex or xelatex
  \ifxetex
    \usepackage{mathspec}
  \else
    \usepackage{fontspec}
  \fi
  \defaultfontfeatures{Ligatures=TeX,Scale=MatchLowercase}
\fi
% use upquote if available, for straight quotes in verbatim environments
\IfFileExists{upquote.sty}{\usepackage{upquote}}{}
% use microtype if available
\IfFileExists{microtype.sty}{%
\usepackage{microtype}
\UseMicrotypeSet[protrusion]{basicmath} % disable protrusion for tt fonts
}{}
\usepackage[margin=1in]{geometry}
\usepackage{hyperref}
\hypersetup{unicode=true,
            pdftitle={Comparative Methods},
            pdfauthor={Brian O'Meara},
            pdfborder={0 0 0},
            breaklinks=true}
\urlstyle{same}  % don't use monospace font for urls
\usepackage{natbib}
\bibliographystyle{apalike}
\usepackage{longtable,booktabs}
\usepackage{graphicx,grffile}
\makeatletter
\def\maxwidth{\ifdim\Gin@nat@width>\linewidth\linewidth\else\Gin@nat@width\fi}
\def\maxheight{\ifdim\Gin@nat@height>\textheight\textheight\else\Gin@nat@height\fi}
\makeatother
% Scale images if necessary, so that they will not overflow the page
% margins by default, and it is still possible to overwrite the defaults
% using explicit options in \includegraphics[width, height, ...]{}
\setkeys{Gin}{width=\maxwidth,height=\maxheight,keepaspectratio}
\IfFileExists{parskip.sty}{%
\usepackage{parskip}
}{% else
\setlength{\parindent}{0pt}
\setlength{\parskip}{6pt plus 2pt minus 1pt}
}
\setlength{\emergencystretch}{3em}  % prevent overfull lines
\providecommand{\tightlist}{%
  \setlength{\itemsep}{0pt}\setlength{\parskip}{0pt}}
\setcounter{secnumdepth}{5}
% Redefines (sub)paragraphs to behave more like sections
\ifx\paragraph\undefined\else
\let\oldparagraph\paragraph
\renewcommand{\paragraph}[1]{\oldparagraph{#1}\mbox{}}
\fi
\ifx\subparagraph\undefined\else
\let\oldsubparagraph\subparagraph
\renewcommand{\subparagraph}[1]{\oldsubparagraph{#1}\mbox{}}
\fi

%%% Use protect on footnotes to avoid problems with footnotes in titles
\let\rmarkdownfootnote\footnote%
\def\footnote{\protect\rmarkdownfootnote}

%%% Change title format to be more compact
\usepackage{titling}

% Create subtitle command for use in maketitle
\newcommand{\subtitle}[1]{
  \posttitle{
    \begin{center}\large#1\end{center}
    }
}

\setlength{\droptitle}{-2em}
  \title{Comparative Methods}
  \pretitle{\vspace{\droptitle}\centering\huge}
  \posttitle{\par}
  \author{Brian O'Meara}
  \preauthor{\centering\large\emph}
  \postauthor{\par}
  \predate{\centering\large\emph}
  \postdate{\par}
  \date{2017-02-17}

\usepackage{booktabs}
\usepackage{makeidx}
\makeindex

\usepackage{amsthm}
\newtheorem{theorem}{Theorem}[chapter]
\newtheorem{lemma}{Lemma}[chapter]
\theoremstyle{definition}
\newtheorem{definition}{Definition}[chapter]
\newtheorem{corollary}{Corollary}[chapter]
\newtheorem{proposition}{Proposition}[chapter]
\theoremstyle{definition}
\newtheorem{example}{Example}[chapter]
\theoremstyle{remark}
\newtheorem*{remark}{Remark}
\begin{document}
\maketitle

{
\setcounter{tocdepth}{1}
\tableofcontents
}
\chapter{Introduction}\label{introduction}

\chapter{First steps}\label{first-steps}

\chapter{Getting data and trees into
R}\label{getting-data-and-trees-into-r}

\chapter{Visualizing data before use}\label{visualizing-data-before-use}

\chapter{Dull model testing}\label{dull-model-testing}

\chapter{Continuous traits}\label{continuous-traits}

\chapter{Brownian Motion and
Correlations}\label{brownian-motion-and-correlations}

\chapter{Discrete Traits}\label{discrete-traits}

\section{Objectives}\label{objectives}

By the end of this chapter, you will:

\begin{itemize}
\tightlist
\item
  Understand how to incorporate rate heterogeneity in discrete trait
  models
\item
  Be able to explain how to test hypotheses about univariate trait
  evolution.
\end{itemize}

Many traits can be thought of as discrete traits: a DNA site comes in
ATGC, protein have one of 20 amino acids, some animals have functional
eyes and others do not, some plants are woody and others are herbaceous.
This is nearly always an approximation. Think of something like limbs:
they seem distinct enough that we even name some groups by their count:
tetrapods, hexapods. Except that when we look closely enough, it becomes
fuzzy: insect mouthparts are derived from limbs, for example, so should
we count these highly modified limbs as limbs (and if not, where in
evolution have they become sufficiently modified to no longer count? And
are nymphalid butterflies tetrapods under that definition yet?). Are
modern whales thought to have four limbs, even though two are extremely
vestigial? Often for neontologists problematic organisms with
intermediate counts are conveniently extinct (so long,
\emph{Basilosaurus}), so we can ignore this fuzziness, but it is often
there (and paleontologists are confronted with it more often). Think
about the details of a species changing from one discrete state to
another, even for something like a seemingly perfectly discrete
character like a base changing from an A to a T. At first this is
present in just a single individual (for a multicellular diploid, on one
DNA strand in one cell in the germ line). Even if under selection, it
will take generations to sweep through to fixation: during that time,
what is ``the'' state of the species? It is even harder to discretize
characters like woodiness (how much wood is required?), eyes (when does
a fish population evolving in a cave finally ``lose'' its eyes?),
biogeography (how finely do you divide the range: by continent? biome?
state?), and so forth.

As for many decisions, this comes back to the biological hypotheses
being tested and the size of the study. For example, one question could
be does a complex trait like wings ever re-evolve once lost?
\citet{Whiting2003} examined this in stick insects: some species have
wings in both sexes, some in one only, and some lack wings in both
sexes. If the question hinges on whether loss of wing genes in a species
prevents re-evolution, then as long as one sex in a species has wings
the species should be coded as having wings. If the question hinges on
the effect of loss of wings on ability to settle new areas, it could be
that having either sex lack wings is enough to prevent effective
colonization, and thus a species with only one sex with wings should be
coded as being wingless. If the study system is large enough to have
sufficient power, one could code this as a four state character,
instead: \textbf{A}: both males and females have wings; \textbf{B}:
males have, females lack wings; \textbf{C}: males lack, females have
wings; and \textbf{D}: both males and females lack wings.

One way to deal with this is to gather discrete data as finely divided
as reasonable and then aggregate. For example, in the stick insect
example, code it as a four state character as above and then, depending
on the biological question, group them. If the question is whether wings
can reappear after being entirely lost, for example, one would group
\textbf{A}, \textbf{B}, and \textbf{C} as having wings (in at least some
members of the species, so the genes remain under selection for
functionality) and \textbf{D} as wingless, but for the dispersal
question one could lump \textbf{B}, \textbf{C}, and \textbf{D}, leaving
\textbf{A} as the other state, or even lump \textbf{B} and \textbf{C}
only.

\begin{tabular}{l|l|l|r|r|r}
\hline
males & females & four\_states & question\_1 & question\_2a & question\_2b\\
\hline
wings & wings & A & 1 & 1 & 2\\
\hline
wings & wingless & B & 0 & 1 & 1\\
\hline
wingless & wings & C & 0 & 1 & 1\\
\hline
wingless & wingless & D & 0 & 0 & 0\\
\hline
\end{tabular}

But, let's assume we can discretize traits and carry on. The simplest
discretization is binary: 0 or 1, often absence or presence (but could
be yellow or blue, etc.). Most models are like our commonly used DNA
models: continuous time with discrete changes, using the same rates
throughout. It is like a model for when an autonomous car will have an
accident: assuming the car works perfectly (gives a whole new meaning to
``blue screen of death'') there's still a chance that at some point a
human is going to run into it. There's a per hour chance of an accident:
let's assume in each hour there's a 0.03\% chance of our autonomous car
having an accident (very roughly based on Google's experience, assuming
a 40 MPH average speed). So the probability of having no accident in the
first hour of driving is 99.97\%; the probability of having no accidents
in the first 40 hours of driving is 99.97\% \^{} 40 = 98.8\%. The number
of accidents is Poisson-distributed; the wait time between accidents is
exponentially distributed. This is the model commonly used in
phylogenetics for discrete traits, though sometimes with more
complexity: one could move (with some rate) between two different rates,
as in a covarion model, for example. A very different model is
Felsenstein's threshold model, which we will discuss in a few weeks. For
now, though, just envision models with a fixed rate of change between
states as long as other characters don't change; it's possible, though,
that the state of other characters do affect these rates (which is what
correlation tests investigate). For example, the probability of
switching from clawed feet to flippers for forelimbs is probably much
higher for species that live in water than on land.

(btw, note the spelling here: having one, two or eight eyes is a
\emph{discrete} trait: individually separate and distinct. Forming an
enclosed bower for hidden mating is a \emph{discreet} trait. The former
is generally far more biologically relevant)

\chapter{Diversification}\label{diversification}

\chapter{SSE methods}\label{sse-methods}

\chapter{RAxML}\label{raxml}

\chapter{Gene Tree Species Tree}\label{gene-tree-species-tree}

\chapter{Dating}\label{dating}

\chapter{Build a method}\label{build-a-method}

\chapter{Placeholder}\label{placeholder}

\chapter{Appendix}\label{appendix}

\bibliography{packages.bib,book.bib}


\end{document}
