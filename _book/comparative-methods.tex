\documentclass[]{book}
\usepackage{lmodern}
\usepackage{amssymb,amsmath}
\usepackage{ifxetex,ifluatex}
\usepackage{fixltx2e} % provides \textsubscript
\ifnum 0\ifxetex 1\fi\ifluatex 1\fi=0 % if pdftex
  \usepackage[T1]{fontenc}
  \usepackage[utf8]{inputenc}
\else % if luatex or xelatex
  \ifxetex
    \usepackage{mathspec}
  \else
    \usepackage{fontspec}
  \fi
  \defaultfontfeatures{Ligatures=TeX,Scale=MatchLowercase}
\fi
% use upquote if available, for straight quotes in verbatim environments
\IfFileExists{upquote.sty}{\usepackage{upquote}}{}
% use microtype if available
\IfFileExists{microtype.sty}{%
\usepackage{microtype}
\UseMicrotypeSet[protrusion]{basicmath} % disable protrusion for tt fonts
}{}
\usepackage[margin=1in]{geometry}
\usepackage{hyperref}
\hypersetup{unicode=true,
            pdftitle={Comparative Methods},
            pdfauthor={Brian O'Meara},
            pdfborder={0 0 0},
            breaklinks=true}
\urlstyle{same}  % don't use monospace font for urls
\usepackage{natbib}
\bibliographystyle{apalike}
\usepackage{longtable,booktabs}
\usepackage{graphicx,grffile}
\makeatletter
\def\maxwidth{\ifdim\Gin@nat@width>\linewidth\linewidth\else\Gin@nat@width\fi}
\def\maxheight{\ifdim\Gin@nat@height>\textheight\textheight\else\Gin@nat@height\fi}
\makeatother
% Scale images if necessary, so that they will not overflow the page
% margins by default, and it is still possible to overwrite the defaults
% using explicit options in \includegraphics[width, height, ...]{}
\setkeys{Gin}{width=\maxwidth,height=\maxheight,keepaspectratio}
\IfFileExists{parskip.sty}{%
\usepackage{parskip}
}{% else
\setlength{\parindent}{0pt}
\setlength{\parskip}{6pt plus 2pt minus 1pt}
}
\setlength{\emergencystretch}{3em}  % prevent overfull lines
\providecommand{\tightlist}{%
  \setlength{\itemsep}{0pt}\setlength{\parskip}{0pt}}
\setcounter{secnumdepth}{5}
% Redefines (sub)paragraphs to behave more like sections
\ifx\paragraph\undefined\else
\let\oldparagraph\paragraph
\renewcommand{\paragraph}[1]{\oldparagraph{#1}\mbox{}}
\fi
\ifx\subparagraph\undefined\else
\let\oldsubparagraph\subparagraph
\renewcommand{\subparagraph}[1]{\oldsubparagraph{#1}\mbox{}}
\fi

%%% Use protect on footnotes to avoid problems with footnotes in titles
\let\rmarkdownfootnote\footnote%
\def\footnote{\protect\rmarkdownfootnote}

%%% Change title format to be more compact
\usepackage{titling}

% Create subtitle command for use in maketitle
\newcommand{\subtitle}[1]{
  \posttitle{
    \begin{center}\large#1\end{center}
    }
}

\setlength{\droptitle}{-2em}
  \title{Comparative Methods}
  \pretitle{\vspace{\droptitle}\centering\huge}
  \posttitle{\par}
  \author{Brian O'Meara}
  \preauthor{\centering\large\emph}
  \postauthor{\par}
  \predate{\centering\large\emph}
  \postdate{\par}
  \date{2017-04-05}

\usepackage{booktabs}
\usepackage{makeidx}
\makeindex

\usepackage{amsthm}
\newtheorem{theorem}{Theorem}[chapter]
\newtheorem{lemma}{Lemma}[chapter]
\theoremstyle{definition}
\newtheorem{definition}{Definition}[chapter]
\newtheorem{corollary}{Corollary}[chapter]
\newtheorem{proposition}{Proposition}[chapter]
\theoremstyle{definition}
\newtheorem{example}{Example}[chapter]
\theoremstyle{remark}
\newtheorem*{remark}{Remark}
\begin{document}
\maketitle

{
\setcounter{tocdepth}{1}
\tableofcontents
}
\chapter{Introduction}\label{introduction}

\chapter{First steps}\label{first-steps}

\chapter{Getting data and trees into
R}\label{getting-data-and-trees-into-r}

\chapter{Visualizing data before use}\label{visualizing-data-before-use}

\chapter{Dull model testing}\label{dull-model-testing}

\chapter{Testing models and methods}\label{testing-models-and-methods}

\chapter{Testing methods}\label{testing-methods}

\subsection{Objectives}\label{objectives}

\begin{itemize}
\tightlist
\item
  Identify and avoid pitfalls in evaluating methods
\item
  Be able to identify methods that have been tested well.
\end{itemize}

\subsection{Kinds of testing}\label{kinds-of-testing}

There are two kinds of testing. One can test the software to make sure
it works properly. If you are trying to calculate the average of a set
of observations, are you using \texttt{mean} or incorrectly using
\texttt{median}? Does it use all the data or does it drop anything past
the fifth observation? For this kind of question, it can be helpful to
do test driven development: write a test, then write code, and
automatically check the code to see if it passes the test. Then, as you
change code, you can rerun all the old tests to verify they still work.
This is often known as unit testing.

But even if software has correctly implemented a method, a more
compelling question is whether the method itself is any good. This comes
down to a few questions:

\subsection{Type I error}\label{type-i-error}

This is when a model incorrectly rejects a true null hypothesis. For
example, do clade A and clade B have exactly the same rate of evolution?
If the truth is that they do, rejecting that to say they are unequal is
a type I error. To test this property, data are ofen simulated under the
null, analyzed under the null and alternate hypotheses, and the
proportion of times the null is incorrectly rejected noted. For a
typical significance theshold of 0.05, this should be 5\% of the time.

This is a major focus\ldots{}

\subsection{Type II error}\label{type-ii-error}

This is incorrectly accepting a false null.

\subsection{Getting rid of typological
thinking}\label{getting-rid-of-typological-thinking}

In biology, typological thinking is bad: one of Darwin's great insights
was that there is substantial variation in nature. However, our
statistical thinking is often limited (see also chapter on dull
hypothesis testing). Appropriate Type I error rates is a nice property,
but how often is the null \emph{actually} true? Never.

\chapter{Continuous traits}\label{continuous-traits}

\chapter{Brownian Motion and
Correlations}\label{brownian-motion-and-correlations}

\chapter{Discrete Traits}\label{discrete-traits}

\chapter{Diversification}\label{diversification}

\chapter{SSE methods}\label{sse-methods}

\chapter{RAxML}\label{raxml}

\chapter{Gene Tree Species Tree}\label{gene-tree-species-tree}

\chapter{Dating}\label{dating}

\chapter{Placeholder}\label{placeholder}

\chapter{Appendix}\label{appendix}

\bibliography{packages.bib,book.bib}


\end{document}
